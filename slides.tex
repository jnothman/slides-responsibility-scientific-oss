\documentclass[aspectratio=169, 22pt]{beamer}

\usepackage{amsmath}
\usepackage{slidedefs}

\title{Open-Source Software's Responsibility to Science}
\subtitle{}
\date{20th July 2018}
\author{Joel Nothman}

\usetheme{usyd-logobar}

\titlegraphic{USYDTitle}
\titlegraphicbackground{usydred}

\newcommand{\hl}{\textcolor{usydred}}

\begin{document}

\titleslide

\section{Gatekeeper}

\begin{points}{Me in open source}
	\p Mostly contributed to popular Scientific Python libraries:\\
	scikit-learn, nltk, scipy.sparse, pandas, ipython, numpydoc
	\p Also information extraction evaluation (neleval), etc.
	\vfill
	\p Community service
	\p ``Volunteer software development''
	\vfill
	\p Caretakers aren't always founders
	\p Founders aren't always caretakers
\end{points}

\begin{centre}{Overheard at ICML}
	\begin{quote}
		\Large
		\it
	Don't worry about how tricky it is to implement \ldots

	\vspace{2em}

	\raggedleft Someone will put it in Scikit-learn and you can just use it.
	\end{quote}
\end{centre}

\begin{points}{Thoughts on an arrogant ML researcher}
	\p Science and engineering rely heavily on open-source \hl{infrastructure}
	\p Scientists think software maintenance is easy
	\p Most users are uncomfortable building their own
	\p Many will only use what's provided in a popular library
	\p Many will not inspect how it works on the inside
	\p Volunteer maintainers act as gatekeepers
\end{points}

\begin{points}{The power of the gatekeeper}
	\p decides \hl{which algorithms} are available
	\p decides how to ensure \hl{correctness} and stability
	\p decides how to \hl{name} or describe the algorithm
	\p decides whether to be \hl{faithful} to a published description
	\p decides on an \hl{API} that may facilitate good science/engineering
\end{points}

\begin{points}{Outline}
	\p blah
\end{points}

\end{document}
